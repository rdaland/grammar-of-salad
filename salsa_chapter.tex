
\chapter{Salsas and Ceviches}

I do not rightly know whether salsas and ceviches deserve to be called salads. On most days, I am inclined to think so. After all, they \begin{itemize}
    \item contain a leafy green (cilantro)
    \item otherwise contain mostly fruits (tomato) and herbs/vegetables (onion, garlic)
    \item have flavor enhanced by acid from a citrus fruit
\end{itemize}
On the other hand, the majority of my friends who disagree point to the following: \begin{itemize}
    \item salsas and ceviches do not contain a majority of leafy green, and cilantro is not especially high in fiber
    \item unlike in most salads, the `dressing' (citric acid) is not normally mixed/blended with an oil/fat
\end{itemize}
Either way, there is a family resemblance among all salsas and ceviches. It all begins with \textit{salsa fresca}, which will be our first recipe.

\newpage \section{Salsa Fresca}

This recipe is the base, the ur-recipe for all of the ceviches. The techniques for this recipe apply to all of the other ones, sometimes in modified form, but this one is the root. The recipe is scaled for a single bunch of cilantro, because that is the smallest amount of fresh cilantro that one can normally buy in American grocery stores. It is worth noting that nothing is sacred when it comes to proportions in salsa -- higher or lower proportions of any ingredient can yield a delicious outcome.

\subsection{Things you need to do beforehand} \begin{itemize}
    \item make sure your tomatoes are ripe -- go to the store several days before you plan to make the salsa unless you're sure you can get ripe ones the day of
    \item if you aren't sure how to get good cilantro, read the cilantro section below first
    \item this recipe is ready to serve the instant you finish mixing -- but it tastes better if you let it sit overnight before you serve
\end{itemize} 

\subsection{Ingredients} \begin{itemize}
    \item 1 bunch cilantro
    \item 3 cloves garlic
    \item 6-10 vine-ripened, Roma tomatoes
    \item 1 large or 2 small white onions
    \item 2 large limes, 4 small limes, or 1/2 cup lime juice
    \item (optional) 2 jalapenos
\end{itemize}

\subsection{Implements} \begin{itemize}
    \item mixing bowl (large enough to hold the raw ingredients without spilling)
    \item cutting board (the tomatoes should fit on it)
    \item cutting knife
\end{itemize}

\subsection{General Flow} \begin{enumerate}
    \item Begin by \textbf{gathering implements}: a large bowl, cutting board, and mid to large sized knife.
    \item \textbf{stem the cilantro}: pluck the leaves by hand, discarding as much stem as possible
    \item \textbf{prep the acid}: cut the limes and squeeze juice into the mixing bowl
    \item \textbf{scoop the tomatoes}, discarding tomato guts and dicing the tomato shells
    \item \textbf{dice the onion}
    \item \textbf{marinate}, ideally for 24--48 hours
\end{enumerate}

Far and away the most tedious (but essential!) step is stemming the cilantro (approx. 1 hr). Depending on your speed, the other prep steps may take around 15 minutes each, for a total preparation time of around 1:45. Personally, I prefer to throw on the latest episode of \textit{Game of Thrones} or listen to James Brown while I prep the cilantro, but to each their own.

\subsection{Detailed instructions for prepping and mixing}

\subsubsection{Cilantro}

There are several stages to handling cilantro that can make a meaningful difference in the quality of your salsa. \begin{enumerate}
    \item buying the cilantro
    \item washing the cilantro
    \item stemming the cilantro
    \item cutting the cilantro
\end{enumerate}
Healthy cilantro will be green throughout. If the bunch has many black spots or yellow colored regions, just don't buy it. It will also be turgid (springy), not limp.

Ideally, you will prep the cilantro within a day of when you buy it. Cilantro goes limp when you refrigerate it, because the fridge sucks the moisture out of it. You can slow this down by tucking the cilantro into water until you're ready to prep it.

Cilantro normally needs to be washed -- it will have a lot of sand and dirt on it. A little bit will probably make it into your dish and that's alright -- but if you don't wash off most of the sand and dirt, it will affect the texture of your salsa. You don't want to bite dirt and hear that grinding dirt sound when you're enjoying a delicious salsa! The most important thing about washing the cilantro is to dry it afterward. If you leave it in a heap after washing, it will start to rot within hours; and if you start to stem and cut it while it is still wet, you'll have a harder time. It is best to either spread it out on a cutting board for air drying, or else roll it up in a paper towel to absorb most of the water. The washing/drying stage does not have to come immediately before stemming and cutting; sometimes I wash and dry the cilantro as soon as I get home from the grocery store, and them stem/cut it the next day. However, once you stem the cilantro, the clock starts ticking, and after it is cut, the clock ticks faster. Your goal should be to get the cilantro into lime juice within an hour of when you stem it.

When you are ready to stem the cilantro, spread out the bunch to the side of your cutting board. Grab a sprig. Often the leaves will come in threes. Pick each leaf by putting it flat between your thumb and forefinger, squeeing gently, and then pulling away from the stem. The goals are to get as little stem as possible, to not bruise the leaf, and to get as much leaf as possible. Put the leaves onto the cutting board, so that they will be ready to cut as soon as you finish stemming. Now do the next spring, and the next, and the next, until they are all gone. This part is very tedious, but it is also important to get the best and most flavor out of your cilantro.

Finally, when you have finished stemming the cilantro, you are almost ready to cut it. For cutting, there are two goals: avoid bruising, and end up with pieces the size you want. There are two important things to avoid bruising. First, before you do any cutting, spread the cilantro leaves out to cover all of your cutting board except the corners and edges. Ideally, you will have a carpet that is not thicker than 2-3 leaves throughout. This is because when you cut through many layers of cilantro, the pressure from leaves above will bruise the leaves below. When there is only 2-3 layers of leaves, you avoid this. Second, when you do the cutting, do it carefully and precisely, with one series of vertical cuts followed by one series of horizontal cuts. Control the spacing between the cuts to get the size of pieces that you want. The reason to do it this way is, it minimizes the number of cuts. The more cutting passes that you make, the more you bruise the cilantro. 

As for what size to aim for, it is mostly up to you. Some people prefer a chunkier salsa, which means a longer marinate time. In this case you may want to space the cuts about half the width of a large leaf. Some people prefer a finer salsa, in which case you may want to space the cuts about 1/4 the width of a large leaf, and need less marinatde time. It is also possible to do `strips' of cilantro by doing fine vertical cuts but coarse horizontal cuts. It is really about the look and feel you want for your salsa. Also, remember to put the cilantro into the lime juice as soon as you cut it. This will pickle the cilantro, preserving it from rotting and extracting the delicious cilantro flavor into your salsa.

\subsubsection{Limes}

The goal here is to get lime juice into the mixing bowl. If you have juice, just pour it. If you are going from limes, there is no special technique. I normally cut them in halves, and then squeeze out the juice between my thumb and finger. A nice trick is to squeeze the juice into your other hand, over the mixing bowl, so that you catch the seeds. I usually prep the acid before I cut the cilantro, so that I can drop the cut cilantro right into it. 

\subsubsection{Tomatoes}

Normally I start prepping tomatoes as soon as I've gotten the cilantro into the lime juice. But really they can be done in any order. There is even an argument for doing the tomatoes first, as they are mildly acidic, and can begin the same pickling process as the lime juice, but more gently. In any case, there are four steps: \begin{enumerate}
    \item getting good tomatoes to begin with
    \item cut the tomatoes in half and de-stem
    \item scoop out the guts with a spoon
    \item cut to the desired portion size
\end{enumerate}

I tend to use vine-ripened roma tomatoes, but it is possible to make a good salsa with many kinds of tomatoes. The keys are that the tomatoes be ripe when you cut them, and that you get tomatoes with a good amount of both acid and sweetness. The most common (beefsteak) tomatoes that you buy in the grocery store tend to be a bit watery, that is why I prefer romas. I have used heirlooms as well, but those are more sweet and less acid, so they tend to go better in a roasted salsa (coming up next!). The way to tell if a tomato is ripe is that it gives a little under your touch, but still feels firm to the touch. When you buy tomatoes on the vine, they will not all be at the exact same size and ripeness, but they will be close. There is several days margin between when a tomato becomes ripe enough to use in salsa, and when it gets overripe. 

\subsubsection{Onions}

The goal is to get diced onions into the mixing bowl. It is possible to use yellow onions (sweet but pungent), red/purple onions (less sweet, less pungent), or even white onions (very pungent). The ratio of onions to tomato can vary, but I aim for 1 part onion to anywhere between 1 and 2 parts tomato by volume. For 6-8 roma tomatoes, this works out to about 2 medium-sized onions.

To prep the onions, chop off the pointy end and the root end, then cut what remains in half, cutting between the two poles you already chopped. Peel the outer, dry layer. Do a series of horizontal and vertical cuts, with spacing designed to give pieces the size you want. For a chunkier salsa, use larger pieces. To reduce crying, water the cutting board before you start chopping the onions (this reduces the acid fumes that cause tearing).

\subsubsection{Garlic}

Take about 3 cloves from the bunch. Cut off the end which was joining the clove to the bunch. To peel, place the flat of the knife blade on top of the garlic with one hand, and use medium force to smash the other hand down onto the flat of the blade, which will partially mash the garlic. The garlic will unpeel easily after this. Repat the process with the other two cloves. Discard the skin and the ends; fine-chop the cloves. For chunkier salsa, use a coarser chop.

\subsubsection{Jalapeno}

Although the flesh of jalapeno is spicy, most of the spice is concentrated in the seeds. The safest way to control the spiciness level is to get rid of as many seeds as possible, and then control the volume of jalapeno flesh. Two large jalapenos will give moderate spice for the proportions in this recipe, but you can add more or use less if you like.

To prep the jalapeno, first cut off the top (the stem end and a little flesh). When you pull the top off, with any luck it will take most of the seeds. But any way, when you peer down into the jalapeno, you should see two or three white strips, possibly with some seeds hanging on them. Cut the jalapeno in half, and then use your knife edge to skim off those white sections, separating all the seeds from the flesh. Now chop the halves into whatever size you desire (smaller for a more even texture and shorter marinate time, larger pieces for a textured salsa and a longer marinate time.

\subsubsection{Mixing}

As you finish prepping each of the ingredients above, slide it off your cutting board into the mixing bowl. (Use the back of your knife, rather than the edge, to avoid dulling the edge unnecessarily.) When you have finally finished chopping everything, it is time to mix. There is no special art to it, just use your hands or mixing implements to mix everything around. Try to mix gently, so that you don't bruise the tomatoes. Do make sure that you scoop from the bottom, in order to distribute the cilantro evenly throughout the salsa.

After you've finished mixing, taste it. Make sure the balance of acid is reasonable -- the salsa should taste bright, but not sour. The garlic should impart an earthy flavor without being overwhelming. If you like, add salt or more jalapeno. But a little salt goes a long way, especially because people eat salsa with chips -- and most chips already have a lot of salt. You can always add salt later, but you can never take it away, so it is better to err on the side of caution.

When you have finished mixing, pour the salsa into your serving bowl. Now it is ready to serve. Or better yet, cover it and put it in the fridge overnight. The lime juice will chemically cook the other ingredients, and the flavors will seep out and intermingle, giving a subtler and richer flavor profile. In fact, it is okay to let this recipe marinate for several days, or even a week, although you will have gotten most of the flavor benefits within 3 days. The lime juice (and natural acid from tomatoes) partially pickles the salsa, protecting it against rot. If you keep the salsa well-refrigerated, you can expect it to last around two weeks.

\newpage  \subsection{Roasted Garlic Salsa}

\newpage  \subsection{Carrot Habanero Salsa}

\newpage \section{Ceviches}

\newpage \section{Fruit Salad}



