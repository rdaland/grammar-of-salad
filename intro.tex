\documentclass{article}

\title{The Grammar of Salad: A cookbook for people who hate recipes}
\author{Robert Daland}

\begin{document}
    \pagenumbering{gobble}
    \maketitle
    \newpage
    \pagenumbering{arabic}
    
I am a linguist, and I am a cook. This book is the result of putting those two together.

The title may need some explanation. Within linguistics, grammar refers to

\begin{itemize}
    \item the (largely unconscious) knowledge that speakers have of how to put together words, phrases, and sentences
    \item theoretical descriptions of that knowledge
\end{itemize}

An example of grammatical knowledge is adjective order. In most or all varieties of English, you put size adjectives before color adjectives. Thus, we would never say yellow big bird, because we would instead say big yellow bird. Some kinds of grammatical knowledge are innate (e.g. the existence of nouns), and some kinds of grammatical knowledge are learned (e.g. adjective ordering). But in either case, a language is a cultural product which is shaped and shared by many people. Linguistics is about identifying the generalizations that underlie that cultural product.

Cooking is also a cultural product which is shaped and shared by many people. So when I talk about "the grammar of salad", I mean the generalizations that apply to salad, rather than the particulars of any one salad or recipe.

This will be a stretch, so work with me for a moment. I have been around the world, and eaten dishes from many different cultures. I continually marvel at the little differences between cultures in how they prepare and eat things. And yet I also see profound commonalities. Vietnamese banh mi is remarkably similar to a Lousiana Po' Boy. Polish pyrogi is identical in many ways to jiaozi (Chinese dumplings), gyoza (Japanese dumplings), and mandu (Korean dumplings). German sauerkraut is at heart the same as Korean cabbage kimchi, except that Koreans tend to use more spice. In linguistics, we use the term typology to refer to the study of similarities and differences between languages. Here, I am interested in the typology of salad.

What makes a salad a salad? What are the properties that tend to be shared by salads across the cuisines of the world? Are there 'clusters' of properties that tend to be shared? For salads which buck the trend (and yet are tasty), why does bucking the trend work? What are the limits, beyond which you can no longer call a dish a salad?

The book begins with the cuisine of Central and Southern America, and more specifically with salsa and ceviche. As I write this Introduction, I don't rightly know whether salsa/ceviche is actually a type of salad or not. That is why it is an excellent place to begin our inquiry. It is also an excellent place to begin the cookbook, because salsa and ceviche are delicious!

The cookbook is organized as follows: \begin{itemize}
    \item Chapter 1: The grammar of salsa
\end{itemize}

To be continued...

\end{document}
