\chapter{Preface}

This book started with a question: \textbf{What makes a great salad?} I am a linguist, and I am a cook. This book is the result of combining those backgrounds and perspectives to analyze salad.

\section{What is salad?}

I started researching for this cookbook by asking my friends, "What is salad?" The answer seemed to be "I can't define it, but I know it when I see it" (as Supreme Court Justice Potter Stewart put it in a slightly different context). In other words, I noticed immediately that there was a lack of specific defining properties, and a lot of diversity in what people said. But there also seemed to be clear agreement on certain things. For example, people generally agree that "potato salad" is not salad (even though it has "salad" in its name). And people generally agree that \textit{sunomono} (a Japanese side dish -- lit. `thing in vinegar' ) is a salad. I don't quite know what salad is and isn't, but I will begin this book with a chapter on salsa, because I think salsa is an interesting edge case.

\section{What is grammar?}

Grammar is the largely unconscious knowledge that speakers have of how to put together words, phrases, and sentences. A grammar of a language is an explicit description of that knowledge, especially of the generalizations that speakers have (rather than the specific words that they know).

Typology is the study of how languages are alike, and how they are different. 

\section{What is a grammar of salad?}

\section{What makes a salad great?}

 To answer that question, I knew I needed to start simpler, breaking it down into pieces. First of all, what is salad? 
The title may need some explanation. Within linguistics, grammar refers to

\begin{itemize}
    \item the (largely unconscious) knowledge that speakers have of how to put together words, phrases, and sentences
    \item theoretical descriptions of that knowledge
\end{itemize}

An example of grammatical knowledge is adjective order. In most or all varieties of English, you put size adjectives before color adjectives. Thus, we would never say yellow big bird, because we would instead say big yellow bird. Some kinds of grammatical knowledge are innate (e.g. the existence of nouns), and some kinds of grammatical knowledge are learned (e.g. adjective ordering). But in either case, a language is a cultural product which is shaped and shared by many people. Linguistics is about identifying the generalizations that underlie that cultural product.

Cooking is also a cultural product which is shaped and shared by many people. So when I talk about "the grammar of salad", I mean the generalizations that apply to salad, rather than the particulars of any one salad or recipe.

This will be a stretch, so work with me for a moment. I have been around the world, and eaten dishes from many different cultures. I continually marvel at the little differences between cultures in how they prepare and eat things. And yet I also see profound commonalities. Vietnamese \textit{banh mi} is remarkably similar to a Lousiana Po' Boy. Polish \textit{pyrogi} is identical in many ways to \textit{jiaozi} (Chinese dumplings), \textit{gyoza} (Japanese dumplings), and \textit{mandu} (Korean dumplings). German sauerkraut is at heart the same as Korean cabbage kimchi, except that Koreans tend to use more spice. In linguistics, we use the term \textbf{typology} to refer to the study of similarities and differences between languages. Here, I am interested in the typology of salad.

What makes a salad a salad? What are the properties that tend to be shared by salads across the cuisines of the world? Are there 'clusters' of properties that tend to be shared? For salads which buck the trend (and yet are tasty), why does bucking the trend work? What are the limits, beyond which you can no longer call a dish a salad?

